\chapter{Conclusion and Outlook}

The implemented sandbox application offers a high level of flexibility for testing the visual influence of field functions.
Being able to reproduce the rays' path in the reference view makes it easy for users to understand how the final image is produced.
Through the Lyapunov exponents overlay and the assisting outline mesh, areas of different behavior in the field can be spotted easily.
Focusing on comprehension of visual phenomena, alternative visualisations of ray paths have been analyzed.
A reference view with two- or three-dimensional wavefronts in general gives a good idea of how rays contribute to the final image, but are not suitable for all kinds of functions.
The three-dimensional outline mesh has shown to be suitable for functions that affect all parts of the scene in an equal or comparable amount.
For mirages in particular, this mesh technique has been extended to visualize a pixel's ray neighborhood to allow the inspection of areas interesting to the user.
The combination of WebGPU and Rust is very promising for the field of scientific visualisation.
While modern standards such as Vulkan focus on more fine grained control and improved performance, they are generally harder to use than OpenGL.
WebGPU fills this void nicely by abstracting over these modern standards.
Additionally, the prospect of having one code base for many platforms including the web makes WebGPU a suitable successor to OpenGL.

In future iterations of the application, the possibility to not only visualize a pixel's ray neighborhood but also the outline mesh of rays for similarly colored areas could make it easier to understand how specific parts of a mirage come into existence.
A larger library of predefined field functions could lay a good foundation for further experiments, for example, by adding functions for the simulation of black holes.
Finally, nonlinear ray casting is a very computation intensive process, which makes it hard to run the application interactively on devices with less powerful GPUs, such as laptops or tablets.
While the acceleration techniques proposed by Gröller~\cite{grollerNonlinearRayTracing1995} may allow interactive rendering of simple scenes such as the Cornell box on such devices, more complex scenes still require powerful hardware.
Researching the integration of remote rendering techniques as proposed by Lamberti and Sanna~\cite{lambertiStreamingBasedSolutionRemote2007} or Liao et al.~\cite{liaoLiveRenderCloudGaming2016} may thus be of interest.

