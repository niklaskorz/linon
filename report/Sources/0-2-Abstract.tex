\begin{center}
  \textsc{Abstract}
\end{center}
%
\noindent In this advanced software practical, the field of nonlinear ray casting is investigated through the development of a sandbox application.
The technical implementation is based on the memory safe Rust programming language and the upcoming WebGPU graphics standard.
The path of rays inside the sandbox is determined by user defined field functions that are evaluated using Runge-Kutta integration.
Additionally to the nonlinear ray casting view, the path of rays can be visualized as an outline mesh in a linearly rendered reference view.
As scene, the Cornell box is used, which consists of two differently sized cuboids in a cubic room.
Several predefined functions are included from which the user can choose: four mirage functions which simulate continuous refraction in heated air, two translation functions that bend the rays in a certain direction, a rotation function and the Lorenz and Rössler attractors.
The influence of the field on the rays can be controlled through a field weight parameter.
By using Lyaponov exponents, areas of different behavior can be highlighted, either as an overlay on the ray casting view or as an outline mesh in the reference view.

\begin{center}
  \textsc{Zusammenfassung}
\end{center}
%
\selectlanguage{ngerman}
\noindent In diesem Fortgeschrittenenpraktikum wird das Feld des nichtlinearen Ray Castings anhand der Entwicklung einer Sandbox-Anwendung untersucht.
Die technische Umsetzung basiert dabei auf der speichersicheren Programmiersprache Rust und dem kommenden WebGPU-Grafikstandard.
Die Pfade der Rays innerhalb der Sandbox werden durch nutzerdefinierte Feldfunktionen bestimmt, die mit Runge-Kutta-Integration ausgewertet werden.
Zusätzlich zum nichtlinearen Ray-Casting-View können die Raypfade als Outline-Mesh in einem linear gerenderten Referenz-View dargestellt werden.
Als Szene wird die Cornell Box genutzt, welche aus zwei verschiedengroßen Quadern in einem würfelförmigen Raum besteht.
Mehrere vordefinierte Funktionen sind enthalten, aus denen die Nutzer wählen können: vier Fata-Morgana-Funktionen, die kontinuierliche Brechung in heißer Luft simulieren, zwei Translation-Funktionen, die die Rays in eine bestimmte Richtung ablenken, eine Rotationsfunktion sowie Lorenz- und Rössler-Attraktoren.
Der Einfluss des Felds kann dabei durch einen Feldgewichtungsparameter reguliert werden.
Durch den Einsatz von Lyapunov-Exponenten können Bereiche unterschiedlichen Verhaltens hervorgehoben werden, entweder als Overlay auf dem Ray-Casting-View oder als Outline-Mesh auf dem Referenz-View.

\selectlanguage{english}

